\subsection{Problem Domain.}
We have chosen the domain of sustainability for our project. Sustainability is a an umbrella term for projects and practices concerned with meeting present needs without compromising the ability of  future generations to meet theirs. It involves solving environmental, economic, and social issues, focusing on how resources can be managed more responsibly to reduce waste and environmental impact. Our project targets domestic food waste, one of the most persistent and overlooked contributors to unsustainable living. Tackling this issue isn't a simple single solution approach but requires changes in technology and behaviour. Particularly regarding encouraging individuals to make more efficient use of resources; lower their carbon footprint; and develop lasting habits that promote sustainable consumption for the long-term.  
\par

\subsection{Technical challenges within the domain.}
\begin{itemize}
    \item It may be difficult to get the information on a product easily - the barcode may not always contain the expiry date or may be damaged and unreadable (NOTE FOR EDITOR: barcodes don't contain any information useful for us yet, only 2D next gen barcodes, likely going to have to scan the expiry date)
    \item 
\end{itemize}
\par

\subsection{Societal challenges within the domain.}
Society as a whole faces a significant challenge in it's goal to reduce emissions to combat climate change. According to the Food and Agriculture Organization of the United Nations (FAO), agricultural industries emitted approximately 16.2 billion tonnes of CO2 globally in 2022, accounting for 29.7 percent of total greenhouse gas emissions \footnote{Food and Agriculture Organization of the United Nations (FAO) (2024) Greenhouse gas emissions from agrifood systems: Global, regional and country trends, 2000–2022. Rome: FAO. Available at: https://openknowledge.fao.org/handle/20.500.14283/cd3167en
 (Accessed: 22 October 2025).}. Additionally, the UK Waste and Resources Action Programme (WRAP) estimates 10.7 million tonnes of food were wasted in the United Kingdom in 2021 \footnote{House of Commons Library (2024) Food waste in the UK (Research Briefing CBP-7552). London: House of Commons Library. Available at: https://commonslibrary.parliament.uk/research-briefings/cbp-7552
 (Accessed: 22 October 2025).}. Much of the food wasted in the UK results from everyday consumer habits, including forgetting about perishable items and not planning meals effectively around expiry dates. This is a widespread issue arising from the fact that keeping a note of expiry dates is both tedious and time-consuming. Additionally, many students struggle to plan a meal last minute using an ingredient which they have just discovered is near its expiry date or purchase too much food resulting in products expiring before it can be used. 

Currently, there are few widely adopted solutions to this problem, with existing expiry-tracking applications, like Beep, often being inaccessible due to high subscription fees. Amid an ongoing cost-of-living crisis in the UK, many students face tight budgets and limited disposable income. Reducing food waste not only lowers grocery expenses but also contributes to a more sustainable and environmentally responsible lifestyle.  To help prevent food waste students need to be able to keep a track of their foods' expiration dates with reminders as it's approaching without taking too much of their time; to be able to plan meals last minute and in advance with their perishable ingredients so nothing is wasted; make smarter decisions when purchasing food.
\par

\subsection{Motivation for the project.}
As busy students, it is difficult to keep track of food expiration dates and to plan meals accordingly. This leads to a significant amount of food being wasted every week, which is not only economically costly but also environmentally irresponsible. We feel that as sustainability-focused young people it is our responsibility to develop a solution that reduces food waste and the associated emissions. Too often, short-shelf-life items are forgotten at the back of the fridge, only to be thrown away days later. By developing a smart expiry-date tracking system, we aim to help students save money, reduce waste, and make more sustainable consumption choices. We are motivated to solve this problem as it affects us personally, and many of us would make use of the solution ourselves. 
\par

\subsection{Novel aspects of the project. }
While other solutions tackling this problem do exist, they have issues that we intend to solve. Namely:
\begin{itemize}
    \item Paid subscription models and low ratings for key features in existing solutions, examples being Plan to Eat\footnote{https://www.plantoeat.com} and Remy\footnote{https://www.remyapp.io} among others 
    \item Disjoint features from several apps that could be unified under one solution
    \item User Experience (\textbf{ELABORATE})
\end{itemize}