\subsubsection{Development Process}
Since we are not developing safety critical software, we can use a more agile approach
to developing software, which ensures that we are meeting stakeholder requirements and prioritizing
the most important features. This also allows us to present more concrete software for feedback.
\par 

\subsubsection{Tools}
We will be using Github to track our changes, allowing us to work simultaneously, 
track issues, and keep track of who is working on what. 
Github's CI/CD will also be used to detect any errors added from commited code.
\par
For scheduling we could use a kanban board to track what tasks need completing and seeing
progress for current tasks. Alternatively we could use Github's issue section to manage
the activities needing completion. This comes at the drawback of being harder to track progress
for.
\subsubsection{Language / Libraries}

For developing Android applications, we can choose from Python using Kivy or Java using the Android SDK.
Python's Kivy module looks to have good documentation
\footnote{https://kivy.org/doc/stable/api-kivy.html}.
For Java, there are two possible choices for documentation:
\begin{enumerate}
  \item Using the official Android site
    \footnote{https://developer.android.com/reference/android/package-summary}
  \item Devdoc provides a simpler alternative
    \footnote{https://devdoc.net/android/Android-r15/reference/android/package-summary.html}
\end{enumerate}
\par
For libraries used, we will prioritize using popular well-known libraries, as they will have more
active security development reducing the likelihood of vulnerabilities. We will focus on using
open-source libraries which gives us the freedom to modify source code to adjust functionality to
better suit our needs, and to reduce cost, ensuring our product is available for our stakeholders.

\subsubsection{Testing}
We will be using Github's CI/CD tool to perform automatic regression testing for our software.
This ensures any added code doesn't break previous expectations.
Furthermore, for local development, we will integrate LSPs and formatting tools to quickly detect
simple bugs, as well as to ensure that the code has a consistent style to make navigating and understanding easier.
Beyond that, we will also ensure new features have appropriate tests added, as well as ensuring commits are
reviewed by at least one other person before merging feature to main.
\subsubsection{Evaluation}
\subsubsection{Work Distribution}
\subsubsection{Development Process}
Since we are not developing safety critical software, we can use a more agile approach
to developing software, which ensures that we are meeting stakeholder requirements and prioritizing
the most important features. This also allows us to present more concrete software for feedback.
\par 

\subsubsection{Tools}
We will be using Github to track our changes, allowing us to work simultaneously, 
track issues, and keep track of who is working on what. 
Github's CI/CD will also be used to detect any errors added from commited code.
\par
For scheduling we could use a ka\subsubsection{Development Process}
Since we are not developing safety critical software, we can use a more agile approach
to developing software, which ensures that we are meeting stakeholder requirements and prioritizing
the most important features. This also allows us to present more concrete software for feedback.
\par 
\subsubsection{Tools}
We will be using Github to track our changes, allowing us to work simultaneously, 
track issues, and keep track of who is working on what. 
Github's CI/CD will also be used to detect any errors added from commited code.
\par
For scheduling we could use a kanban board to track what tasks need completing and seeing
progress for current tasks. Alternatively we could use Github's issue section to manage
the activities needing completion. This comes at the drawback of being harder to track progress
for.
\subsubsection{Language / Libraries}
For developing Android applications, we can choose from Python using Kivy or Java using the Android SDK.
Python's Kivy module looks to have good documentation
\footnote{https://kivy.org/doc/stable/api-kivy.html}.
For Java, there are two possible choices for documentation:
\begin{enumerate}
  \item Using the official Android site
    \footnote{https://developer.android.com/reference/android/package-summary}
  \item Devdoc provides a simpler alternative
    \footnote{https://devdoc.net/android/Android-r15/reference/android/package-summary.html}
\end{enumerate}
\par
For libraries used, we will prioritize using popular well-known libraries, as they will have more
active security development reducing the likelihood of vulnerabilities. We will focus on using
open-source libraries which gives us the freedom to modify source code to adjust functionality to
better suit our needs, and to reduce cost, ensuring our product is available for our stakeholders.
\subsubsection{Testing}
We will be using Github's CI/CD tool to perform automatic regression testing for our software.
This ensures any added code doesn't break previous expectations.
Furthermore, for local development, we will integrate LSPs and formatting tools to quickly detect
simple bugs, as well as to ensure that the code has a consistent style to make navigating and understanding easier.
Beyond that, we will also ensure new features have appropriate tests added, as well as ensuring commits are
reviewed by at least one other person before merging feature to main.
\subsubsection{Evaluation}
\subsubsection{Work Distribution}
nban board to track what tasks need completing and seeing
progress for current tasks. Alternatively we could use Github's issue section to manage
the activities needing completion. This comes at the drawback of being harder to track progress
for.
\subsubsection{Language / Libraries}

For developing Android applications, we can choose from Python using Kivy or Java using the Android SDK.
Python's Kivy module looks to have good documentation
\footnote{https://kivy.org/doc/stable/api-kivy.html}.
For Java, there are two possible choices for documentation:
\begin{enumerate}
  \item Using the official Android site
    \footnote{https://developer.android.com/reference/android/package-summary}
  \item Devdoc provides a simpler alternative
    \footnote{https://devdoc.net/android/Android-r15/reference/android/package-summary.html}
\end{enumerate}
\par
For libraries used, we will prioritize using popular well-known libraries, as they will have more
active security development reducing the likelihood of vulnerabilities. We will focus on using
open-source libraries which gives us the freedom to modify source code to adjust functionality to
better suit our needs, and to reduce cost, ensuring our product is available for our stakeholders.

\subsubsection{Testing}
We will be using Github's CI/CD tool to perform automatic regression testing for our software.
This ensures any added code doesn't break previous expectations.
Furthermore, for local development, we will integrate LSPs and formatting tools to quickly detect
simple bugs, as well as to ensure that the code has a consistent style to make navigating and understanding easier.
Beyond that, we will also ensure new features have appropriate tests added, as well as ensuring commits are
reviewed by at least one other person before merging feature to main.
\subsubsection{Evaluation}
\subsubsection{Work Distribution}
